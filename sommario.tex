\addcontentsline{toc}{chapter}{Summary}
\begin{abstract}
\thispagestyle{plain}
Gli andamenti globali indicano che il numero di utilizzatori di internet nel mondo ha ormai raggiunto la cifra di 4.54 miliardi di persone, con un incremento del 7 per cento (di cui 298 milioni di nuovi utenti) da gennaio 2019 fino a gennaio 2020. Inoltre, a gennaio 2020 sono stati registrati 3.8 miliardi di utilizzatori di social media, cifra che ha avuto un incremento del +9\% (321 nuovi utenti) rispetto allo stesso mese dell'anno precedente. In totale, più di 5.19 miliardi di persone utilizzano al giorno d'oggi telefoni cellulari, avendo avuto un incremento di 124 milioni (2.4\%) di nuovi utilizzatori rispetto all'anno precedente.
% \footnote{https://datareportal.com/reports/digital-2020-global-digital-overview}

L'impatto dei sopra citati dati hanno dato una grossa spinta alla crescita di un altro settore, quello delle piattaforme digitali. Una piattaforma digitale può essere definita come "un blocco fondamentale che provvedere una funzione essenziale ad un sistema tecnologico e serve da fondamento sopra il quale altri prodotti complementari, tecnologie e servizi possono essere costruiti".
% \cite{digital_platform_def}

Le industrie che hanno fatto leva sulle piattafrome digitali hanno raggiunto una significante crescita di dimensioni e scala. Per esempio, le piattaforme digitali che operano nelle aree dell'e-commerce e dello sviluppo software hanno superato i 700 miliardi di dollari di valore di mercato. In aggiunta la crescita delle piattaforme digitali ha trasformato il paesaggio di molte industrie come per esempio quella dei trasporti (Uber, Grab), dell'ospitalità (Airbnb, CouchSurfing), e sviluppo software (Apple iOS, Google Android).
% \cite{digital_platform_number}

Questo report riassume come il processo del design di una piattaforma digitale assuma i connotati dello stesso processo di design che ha l'ingegneria del software. Nel primo capitolo vi sarà una breve discussione della definizione delle piattaforme digitali e delle loro tipologie per poter dare forma al contesto nel quale il nostro progetto si sta sviluppando.

Poichè questo progetto è il riassunto di uno stage, ne vedremo gli stadi di design e sviluppo così come le metodologie e tecnologie che sono necessarie per lo sviluppo di una piattaforma digitale quale è stata il nostro progetto.
\end{abstract}